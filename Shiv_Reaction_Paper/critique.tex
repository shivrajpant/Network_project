%!TEX root = paper.tex

%%%
%%% Definitions of concepts necessary for any reader to understand the rest of the paper.
%%%
%%%

\section{Critique} \label{sec:critique}
%\vspace{-0.1in}

\subsection{Paper \cite{aw2018analyzing}} 

Strengths: The authors have discovered interesting results from the analysis of the subject networks. A synthetic model of the network has been created to validate their results.The authors have provided direct applications of such synthetic models in numerical experimentation and simulation. Experimentation is performed rigorously with clear diagrams and plots.
	
Weaknesses: Despite providing a strong analysis, the paper suffers from some drawbacks and limitations. The synthetic model is created for networks like BITCOIN ALPHA. Hence it is questionable to be used for other types of networks. Furthermore, the model is validated against another similar network BITCOIN OTC. Validation against only one network has a threat of validity against other networks. The analysis fails to include experiments against a wide variety of datasets.
\subsection{Paper \cite{kumar2016edge}}
Strengths: One of the strength of the paper is the introduction of two novel metrics applied to nodes of networks to solve the problem of predicting edge weight in weighted signed networks. The authors have mentioned many direct application of such solution with clear examples pertaining to real-life social networks. Furthermore, the solution has been validated against a wide variety of datasets. The approach is novel and rigourously worked out. Another strength is that the datasets and codes have been made publicly available so that any curious reader/researcher can reuse and validate the approach. 

Weaknesses: The study suffers from some weaknesses as follows: The results show that the approach does not perform perfectly in all the datasets studied.e.g. the goodness measure alone performs poor in case of regression model and the authors fail to provide explannation for that behaviour. Another limitation is that the solution has been applied in social networks. All the networks taken for experiments are social networks. Due to lack of experiments with other varieties of datasets, it is not clear if this approach is applicable to other types of networks.
